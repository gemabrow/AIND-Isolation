\documentclass[12pt]{article}

% ams packages
\usepackage{amsthm,amsfonts,amssymb,amsmath}
% addl. packages not requiring options to be passed
\usepackage{enumitem, epsfig, fancyhdr, graphicx, listings,%
            marvosym, multicol, pgf, tikz, times, xcolor}
% other packages with options
\usepackage[margin=2.3cm]{geometry}

% tikz automata settings 
\usetikzlibrary{automata,arrows,positioning}
% set text of transition to initial state to nothing
\tikzset{initial text={}}

% custom commands
\newcommand{\wip}[2][red]{\textcolor{#1}{#2}}
\newcommand{\bigO}[1]{\mathcal{O}(#1)}
\newcommand{\indist}[1]{\equiv_{L}#1}
\newcommand{\mt}[1]{\mathrm{#1}}
\newcommand{\opt}[1]{$\mathrm{OPT}(#1)$}
\newcommand{\sopt}[2]{$\mathrm{OPT_{#1}}(#2)$}

% layout settings
\renewcommand{\footrulewidth}{0.5pt}
\setlength\parindent{0pt}
\setlength{\columnsep}{3cm}

% listings options
\lstset{
  basicstyle=\itshape,
  xleftmargin=3em,
  literate={->}{$\rightarrow$}{2}
           {lambda}{$\lambda$}{1}
           {epsilon}{$\varepsilon$}{1}
           {times}{$\times$}{1}
           {start}{$S_{0}$}{1}
}

% header settings
\pagestyle{fancy}
\lhead{AI-NAND: Heuristic Analysis}
\rhead{Gerald Brown}%ID:1377271}}
\cfoot{}
\rfoot{\thepage}

\begin{document}
Hello
% \section*{Exercises}
% \begin{enumerate}[label={2.\theenumi}]
%     \setcounter{enumi}{10} 
%     \item {\bf Convert the given CFG to an equivalent PDA using Theorem 2.20:}
%         \begin{lstlisting}
%         E -> E + T | T
%         T -> T times F | F
%         F -> (E) | a
%         \end{lstlisting}
%         
%         Informally, we can describe the PDA recognizing the above CFG as
%         follows:
%         \begin{enumerate}[label={\arabic*.}]
%             \item Place the \$ symbol on top of the stack, and the (assumed)
%                   start variable $E$ after that.
%             \item Repeat the following until the \$ symbol is again encountered:
%                 \begin{itemize}
%                     \item If the top of the stack is an $E$, pop it and then
%                           push either $E+T$ or $T$ onto the stack in a non-
%                           deterministic way.
%                     \item If the top of the stack is a $T$, pop it and then
%                           push either $T\times F$ or $F$ onto the stack non-
%                           deterministically.
%                     \item If the top of the stack is an $F$, pop it and then
%                           push either $(E)$ or $a$ onto the stack non-
%                           deterministically.
%                     \item If the top of the stack is terminal, read from input
%                           comparing the two symbols -- if they match, repeat,
%                           otherwise reject this branch.
%                 \end{itemize}
%             \item If the \$ symbol is encountered, enter the accept state.
%         \end{enumerate}
%        
%         Formally, this definition is $(Q, \Sigma, \Gamma, \delta, q_0, F)$
%         where $Q=\{q_0, q_1\}$, $\Sigma = \{+,\times, (,),a\}$,
%         $\Gamma = \{E, T, F\}\cup \Sigma$ and $F=\{q_1\}$. Its transition
%         function $\delta: Q\times \Sigma_{\varepsilon} \times \Gamma_{\varepsilon}
%         \rightarrow P(Q\times\Gamma_{\varepsilon})$ is as follows:
%         
%         \[ 
%             \delta(q,x,y) = 
%             \begin{cases}
%                 \{(q_0,E+T), (q_0, T)\} & if q=q_0, x=\varepsilon, y=E \\
%                 \{(q_0,T\times F), (q_0, F)\} & if q=q_0, x=\varepsilon, y=T \\
%                 \{(q_0,(E), (q_0 a)\} & if q=q_0, x=\varepsilon, y=F \\
%                 \{(q_0,\varepsilon)\} & if q=q_0, x=y \\
%                 \{(q_1,\varepsilon)\} & if q=q_0, x=\varepsilon, y=\$
%             \end{cases}
%         \]
% 
%         The transition diagram is depicted in Figure 1.
% 
%         \begin{figure}
%         \centering
%         \begin{tikzpicture}[->,>=stealth,shorten >=1pt,auto,node distance=4cm,
%                     scale = 1, semithick, transform shape]
%         
%           \node[initial,state] (E)                    {$q_{\text{start}}$};
%           \node[state]         (0) [      right of=E] {};
%           \node[state]         (L) [      right of=0] {$q_{\text{loop}}$};
%           \node[state]         (A1)[above       of=E] {};
%           \node[state]         (A2)[      right of=A1]{};
%           \node[state]         (M1)[below       of=E] {};
%           \node[state]         (M2)[below right of=M1]{};
%           \node[state]         (P1)[      right of=M2]{};
%           \node[state]         (P2)[above right of=P1]{};
%           \node[state,accepting](A)[      right of=L] {$q_{\text{accept}}$};
%           \path [every node/.style={sloped,anchor=south,auto=false}]
%               (E) edge              node {$\varepsilon,\ \varepsilon\rightarrow \$$} (0)
%               (0) edge              node {$\varepsilon,\ \varepsilon\rightarrow E$} (L)
%               (L) edge              node {$\varepsilon,\ E\rightarrow T$} (A1)
%                   edge              node {$\varepsilon,\ T\rightarrow F$} (M1)
%                   edge              node {$\varepsilon,\ F\rightarrow )$} (P1)
%                   edge              node {$\varepsilon,\ \$\rightarrow \varepsilon$} (A)
%                   edge [loop above] node {see *} (L)
%               (A1)edge              node {$\varepsilon,\ \varepsilon\rightarrow +$} (A2)
%               (A2)edge              node {$\varepsilon,\ \varepsilon\rightarrow E$} (L)
%               (M1)edge              node {$\varepsilon,\ \varepsilon\rightarrow\times$} (M2)
%               (M2)edge              node {$\varepsilon,\ \varepsilon\rightarrow T$} (L)
%               (P1)edge              node {$\varepsilon,\ \varepsilon\rightarrow E$} (P2)
%               (P2)edge              node {$\varepsilon,\ \varepsilon\rightarrow ($} (L);
%         \end{tikzpicture}
%         
%         \begin{lstlisting}
%         * 
%         epsilon, E -> T
%         epsilon, T -> F
%         epsilon, F -> a
%         (, ( -> epsilon
%         ), ) -> epsilon
%         +, + -> epsilon
%         times, times -> epsilon
%         a, a -> epsilon
%         \end{lstlisting}
%         \caption{The state diagram for 2.11} \label{The state diagram for our PDA}
%         \end{figure}
%         
%     \item {\bf Convert the given CFG to an equivalent PDA using Theorem 2.20:}
%         \begin{lstlisting}
%         R -> XRX | S
%         S -> aTb | bTa 
%         T -> XTX | X | epsilon 
%         X -> a | b 
%         \end{lstlisting}
%         
%         Informally, we can describe the PDA recognizing the above CFG as
%         follows:
%         \begin{enumerate}[label={\arabic*.}]
%             \item Place the \$ symbol on top of the stack, and the (assumed)
%                   start variable $R$ after that.
%             \item Repeat the following until the \$ symbol is again encountered:
%                 \begin{itemize}
%                     \item If the top of the stack is an $R$, pop it and then
%                           push either $XRX$ or $S$ onto the stack in a non-
%                           deterministic way.
%                     \item If the top of the stack is an $S$, pop it and then
%                           push either $aTb$ or $bTa$ onto the stack non-
%                           deterministically.
%                     \item If the top of the stack is a $T$, pop it and then
%                           push either $XTX$, $X$, or $\varepsilon$ onto the
%                           stack non-deterministically.
%                     \item If the top of the stack is an $X$, pop it and then
%                           push either $a$ or $b$ onto the stack non-
%                           deterministically.
%                     \item If the top of the stack is terminal, read from input
%                           comparing the two symbols -- if they match, repeat,
%                           otherwise reject this branch.
%                 \end{itemize}
%             \item If the \$ symbol is encountered, and assuming that all of the
%                   input has been read, enter the accept state.
%         \end{enumerate}
%         
%         Formally, this definition is $(Q, \Sigma, \Gamma, \delta, q_0, F)$
%         where $Q=\{q_0, q_1\}$, $\Sigma = \{a, b\}$,
%         $\Gamma = \{R, S, T, X\}\cup \Sigma$ and $F=\{q_1\}$. Its transition
%         function $\delta: Q\times \Sigma_{\varepsilon} \times \Gamma_{\varepsilon}
%         \rightarrow P(Q\times\Gamma_{\varepsilon})$ is as follows:
%         
%         \[ 
%             \delta(q,x,y) = 
%             \begin{cases}
%                 \{(q_0,XRX), (q_0, S)\} & if q=q_0, x=\varepsilon, y=R \\
%                 \{(q_0,aTb), (q_0, bTa)\} & if q=q_0, x=\varepsilon, y=S \\
%                 \{(q_0,XTX), (q_0, X), (q_0, \varepsilon)\} & if q=q_0, x=\varepsilon, y=T \\
%                 \{(q_0,a), (q_0, b)\} & if q=q_0, x=\varepsilon, y=T \\
%                 \{(q_0,\varepsilon)\} & if q=q_0, x=y \\
%                 \{(q_1,\varepsilon)\} & if q=q_0, x=\varepsilon, y=\$
%             \end{cases}
%         \]
% 
% %        The transition diagram is depicted in Figure 2.
% %
% %        \begin{figure}
% %        \centering
% %        \begin{tikzpicture}[->,>=stealth,shorten >=1pt,auto,node distance=4cm,
% %                    scale = 1, semithick, transform shape]
% %        
% %          \node[initial,state] (R)                    {$q_{\text{start}}$};
% %          \node[state]         (0) [      right of=R] {};
% %          \node[state]         (L) [      right of=0] {$q_{\text{loop}}$};
% %          \node[state]         (A1)[above       of=R] {};
% %          \node[state]         (A2)[      right of=A1]{};
% %          \node[state]         (M1)[below       of=R] {};
% %          \node[state]         (M2)[below right of=M1]{};
% %          \node[state]         (P1)[      right of=M2]{};
% %          \node[state]         (P2)[above right of=P1]{};
% %          \node[state,accepting](A)[      right of=L] {$q_{\text{accept}}$};
% %          \path [every node/.style={sloped,anchor=south,auto=false}]
% %              (R) edge              node {$\varepsilon,\ \varepsilon\rightarrow \$$} (0)
% %              (0) edge              node {$\varepsilon,\ \varepsilon\rightarrow E$} (L)
% %              (L) edge              node {$\varepsilon,\ E\rightarrow T$} (A1)
% %                  edge              node {$\varepsilon,\ T\rightarrow F$} (M1)
% %                  edge              node {$\varepsilon,\ F\rightarrow )$} (P1)
% %                  edge              node {$\varepsilon,\ \$\rightarrow \varepsilon$} (A)
% %                  edge [loop above] node {see *} (L)
% %              (A1)edge              node {$\varepsilon,\ \varepsilon\rightarrow +$} (A2)
% %              (A2)edge              node {$\varepsilon,\ \varepsilon\rightarrow E$} (L)
% %              (M1)edge              node {$\varepsilon,\ \varepsilon\rightarrow\times$} (M2)
% %              (M2)edge              node {$\varepsilon,\ \varepsilon\rightarrow T$} (L)
% %              (P1)edge              node {$\varepsilon,\ \varepsilon\rightarrow E$} (P2)
% %              (P2)edge              node {$\varepsilon,\ \varepsilon\rightarrow ($} (L);
% %        \end{tikzpicture}
% %        
% %        \begin{lstlisting}
% %        * 
% %        epsilon, E -> T
% %        epsilon, T -> F
% %        epsilon, F -> a
% %        (, ( -> epsilon
% %        ), ) -> epsilon
% %        +, + -> epsilon
% %        times, times -> epsilon
% %        a, a -> epsilon
% %        \end{lstlisting}
% %        \caption{The state diagram for our PDA} \label{The state diagram for our PDA}
% %        \end{figure}
%     \setcounter{enumi}{13}
%     \item {\bf Convert the following CFG into an equivalent CFG in Chomsky normal form:
%         \begin{lstlisting}
%         A -> BAB | B | epsilon
%         B -> 00 | epsilon
%         \end{lstlisting}}
%         
%         \begin{lstlisting}
%         S0 -> AB | BA | BD | BB | CC | epsilon 
%         A -> AB | BA | BD | BB | CC 
%         B -> CC
%         C -> 0
%         D -> AB
%         \end{lstlisting}
%         
%     \setcounter{enumi}{29}
%     \item {\bf Use the pumping lemma to prove the following languages are not
%             context-free:}
% 
%         \begin{enumerate}[label={\alph*.},font={\bfseries}]
%             \item $A = \left\{0^n1^n0^n1^n\ |\ n\geq 0\right\}$
% 
%                     \begin{proof}
%                     Let $p$ be the pumping length from the pumping lemma and
%                     $s = 0^p1^p0^p1^p$. We sill show that $s = uvxyz$ cannot
%                     be pumped.
%                     \begin{itemize}
%                         \item If $v$ or $y$ consist of more than a single symbol from
%                     our alphabet, then the order of the symbols appearance in
%                     $uv^2xy^2$ will be out of order and, thusly, not in $A$. \Lightning
%                         \item If $v$ and $y$ each consist of a single symbol
%                         from our alphabet, we instead end up with $uv^2xy^2z$
%                         containing unequal length of substrings of 0s and 1s.
%                         These unequal substrings, however, would result in $s\notin A$. \Lightning
%                     \end{itemize}
% 
%                     By these contradictions showing that $s$ cannot be pumped,
%                     $A$ must not be context-free.
%                     \end{proof} 
%                     
%             \item $B = \left\{0^n\#0^{2n}\#0^{3n}\ |\ n\geq 0\right\}$
% 
%                 \begin{proof}
%                 Suppose $p$ is the pumping length of $B$ and let $s =  0^p\#0^{2p}\#0^{3p}$.
%                 We will show that $s = uvxyz$ cannot be pumped.
%                     \begin{itemize}
%                         \item If $v$ or $y$ contain \# then $xv^2wy^2$ will
%                         contain more than two \#s, resulting in $s\notin B$. \Lightning
%                     \item Segmenting $s$ such that $v,y\in\{0^p, 0^{2p}, 0^{3p}\}$
%                         leaves one of these segments uncontained by $v$ or $y$, resulting
%                         in one segment not being proportianally lengthened.
%                         From this, we would again arrive at $s\notin A$. \Lightning
%                     \end{itemize}
% 
%                 By these contradictions showing that $s$ cannot be pumped,
%                 $B$ must not be context-free.
%                 
%                 \end{proof}
%             \item $C = \left\{w\#t\ |\ w\ \text{is a substring of}\ t, \text{where}\ w,t\in\{a,b\}^*\right\}$
%                 \begin{proof}
%                 Let $p$ be the pumping length of $C$ and let $s =  a^pb^p\#a^pb^p$.
%                 We will show that $s = uvxyz$ cannot be pumped.
%                     \begin{itemize}
%                         \item If $v$ or $y$ contain \# then $xv^0wy^0$ will
%                         contain any \#s, resulting in $s\notin B$. \Lightning
%                         \item If $v$ or $y$ occur on the left side of the \#,
%                         then $uv^2xy^2z$ will have a longer left side. \Lightning
%                         \item If $v$ or $y$ occur on the right side \# then
%                         $xv^0wy^0$ will also be longer on its left side,
%                         resulting in $s\notin C$. \Lightning
%                     \end{itemize}
% 
%                 By these contradictions showing that $s$ cannot be pumped,
%                 $C$ must not be context-free.
%                 
%                 \end{proof}
%              \item $D = \left\{t_1\#t_2\#\cdots \#t_k\ |\ k\geq 2,\ \text{each}\ t_i\in\{a,b\}^*,\ \text{and}\ t_i=t_j\ \text{for some}\ i\neq j\right\}$
% 
%                 \begin{proof}
%                 Let $p$ be the pumping length of $D$ and let $s =  0^p\#0^{2p}\#0^{3p}$.
%                 We will show that $s = uvxyz$ cannot be pumped.
%                     \begin{itemize}
%                         \item If $v$ or $y$ contain \# then $xv^0wy^0$ will
%                         contain any \#s, resulting in $s\notin B$. \Lightning
%                         \item If $v$ or $y$ occur on the left side of the \#,
%                         then $uv^2xy^2z$ will have a longer left side. \Lightning
%                         \item If $v$ or $y$ occur on the right side \# then
%                         $xv^0wy^0$ will also be longer on its left side,
%                         resulting in $s\notin C$. \Lightning
%                     \end{itemize}
% 
%                 By these contradictions showing that $s$ cannot be pumped,
%                 $D$ must not be context-free.
%                 
%                 \end{proof}
%         \end{enumerate}
% %    \item {\bf Let $B$ be the language of all palindromes over $\{0,1\}$ containing equal numbers of
% %               0s and 1s. Show that $B$ is not context free.}
% %        \begin{proof}
% %         
% %        \end{proof} 
% %        
% %        \begin{enumerate}[label={\alph*.},font={\bfseries}]
% %            \setcounter{enumii}{1}
% %            \item $L = \left\{0^m1^n\ |\ m\neq n\right\}$
% %
% %                The language is not regular. We will prove this using the
% %                Myhill-Nerode Theorem.
% %
% %                \begin{proof}
% %                    Consider the set $F = \left\{0^n\ |\ n\geq 1\right\}$ which
% %                    is clearly infinite.
% %
% %                    Let $x$ and $y$ be any two members of $F$ such that $x=0^i$
% %                    and $y=0^k$, where $i\neq j$. 
% %                
% %                    Let $z=1^i$. Now it is clear that $x$ and $y$
% %                    are discernible given that $xz=0^i1^i \notin L$ whereas 
% %                    $yz = 0^k1^i \in L$, implying $L$ is not regular.
% %                \end{proof}
% %
% %            \item $L = \left\{w\ |\ w\in \left\{0,1\right\}^*\ \text{is not a palindrome}\right\}$
% %
% %                Assume, for the sake of contradiction, that the language $L$ is
% %                regular. If $L$ is regular, then the complement of $L$, 
% %                $\bar{L}$, is also regular by the closure of the class of
% %                regular languages under complement. That is,
% %                $\bar{L} = \left\{w\ |\ w\in\left\{0,1\right\}^*\ \text{is a palindrome}\right\}$
% %                is a regular language.
% %
% %
% %        \end{enumerate}
% %\end{enumerate}
% %\begin{itemize}
% %    \item {\bf Prove any language of your choice is regular with the Myhill-Nerode
% %          Theorem. Could you have done the proof (of regularity) with the
% %          Pumping Lemma?}
% %
% %          Let $L=\left\{1^n\ |\ n\ \text{is odd}\right\}$. We will prove $L$ is
% %          regular using the Myhill-Nerode Theorem. 
% %          \begin{proof}
% %              Consider the finite subset $A = \left\{1^n\ |\ n\ \text{is even}\right\}$.
% %          \end{proof}
% %
% %\end{itemize}
% %\begin{enumerate}[label={2.\theenumi}]
% %    \setcounter{enumi}{2}
% %    \item {\bf Answer each part for the following context-free grammar G.} 
% %        \begin{enumerate}[label={\alph*.},font={\bfseries}]
% %            \item $\{R, S, T, X\}$
% %            \item $\{a, b\}$
% %            \item $R$
% %            \item $\{ab, ba, aaba\}$
% %            \item $\{\lambda, a, b\}$
% %            \item False
% %            \item True
% %            \item False
% %            \item True
% %            \item True
% %            \item False
% %            \item True
% %            \item True
% %            \item False
% %            \item $L = \left\{ w\in \{a,b\}^*\ |\ w\ \text{is not a palindrome}\right\}$
% %        \end{enumerate}
% %    \item With $\Sigma = \{0,1\}$, give CFGs generating the given language.
% %        \begin{enumerate}[label={\alph*.},font={\bfseries}]
% %            \item $w$ contains at least three 1s
% %                \begin{lstlisting}
% %                S -> R1R1R1R
% %                R -> 0R | 1R | lambda
% %                \end{lstlisting}
% %            \item $w$ starts and ends with the same symbol
% %                \begin{lstlisting}
% %                S -> 0A | 1B | lambda
% %                A -> AA | BA | 0
% %                B -> BB | AB | 1
% %                \end{lstlisting}
% %            \item $w$ has odd length 
% %                \begin{lstlisting}
% %                S -> R | X
% %                R -> XXR | X  
% %                X -> 0 | 1
% %                \end{lstlisting}
% %            \item $w$ is odd and its middle symbol is 0
% %                \begin{lstlisting}
% %                S -> 0S1 | 1S0 | 1S1 | 0S0 | 0 
% %                \end{lstlisting}
% %         \end{enumerate}
% \end{enumerate}
% %\begin{itemize}
% %    \item {\bf Give an informal description of a PDA that recognizes the
% %        language $\left\{x\in \left\{a,b\right\}^*\ |\ 
% %        x\neq ww\ \text{for some}\ w\in\left\{a,b\right\}^*\right\}$.}
% %        Informally, we can describe the PDA recognizing the above as
% %        follows:
% %        \begin{enumerate}[label={\arabic*.}]
% %            \item Place the \$ symbol on top of the stack, and the (assumed)
% %                  start variable $E$ after that.
% %            \item Repeat the following until the \$ symbol is again encountered:
% %                \begin{itemize}
% %                    \item If the top of the stack is an $E$, pop it and then
% %                          push either $E+T$ or $T$ onto the stack in a non-
% %                          deterministic way.
% %                    \item If the top of the stack is a $T$, pop it and then
% %                          push either $T\times F$ or $F$ onto the stack non-
% %                          deterministically.
% %                    \item If the top of the stack is an $F$, pop it and then
% %                          push either $(E)$ or $a$ onto the stack non-
% %                          deterministically.
% %                    \item If the top of the stack is terminal, read from input
% %                          comparing the two symbols -- if they match, repeat,
% %                          otherwise reject this branch.
% %                \end{itemize}
% %            \item If the \$ symbol is encountered, enter the accept state.
% %        \end{enumerate}
% % 
% %%
% %%        \begin{lstlisting}
% %%        S -> aSa | bSb | B
% %%        B -> bAa | aAb
% %%        A -> lambda | aA | bA
% %%        \end{lstlisting}
% %\end{itemize}
\end{document}
\grid
