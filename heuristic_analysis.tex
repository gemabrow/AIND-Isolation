\documentclass[12pt]{article}

% ams packages
\usepackage{amsthm,amsfonts,amssymb,amsmath}
% addl. packages not requiring options to be passed
\usepackage{enumitem, epsfig, fancyhdr, fancyvrb, graphicx, listings,%
            marvosym, multicol, pgf, tcolorbox, tikz, times, xcolor}
% other packages with options
\usepackage[margin=2.3cm]{geometry}
\usepackage[newfloat]{minted}
\usepackage[parfill]{parskip}
% code style for minted
\usemintedstyle{monokai}
% tikz automata settings 
\usetikzlibrary{automata,arrows,positioning}
% set text of transition to initial state to nothing
\tikzset{initial text={}}

% color definitions
\definecolor{pink}{HTML}{F92672}
\definecolor{teal}{HTML}{6699DF}
\definecolor{purple}{HTML}{AE81FF}
\definecolor{gray}{HTML}{A59F85}
\definecolor{charcoal}{HTML}{383830}
\definecolor{bg-black}{HTML}{272822}

% define `codeblock' environment, default: python
\tcbuselibrary{minted}
\newtcblisting{codeblock}[2][python]{
    colback             =   bg-black,
    colframe            =   charcoal,
    listing only,
    minted options      =   {fontsize=\footnotesize},
    minted language     =   #1,
    title               =   \texttt{\textcolor{gray}{#2}},
}
\setmintedinline{style=friendly}

% custom commands
\newcommand{\wip}[2][red]{\textcolor{#1}{#2}}
\newcommand{\bigO}[1]{\mathcal{O}(#1)}
\newcommand{\indist}[1]{\equiv_{L}#1}
\newcommand{\mt}[1]{\mathrm{#1}}
\newcommand{\opt}[1]{$\mathrm{OPT}(#1)$}
\newcommand{\sopt}[2]{$\mathrm{OPT_{#1}}(#2)$}
\newcommand{\score}[1][]{\texttt{AB\_Custom{#1}}}
\newcommand{\code}[2][python]{\mintinline{#1}{#2}}

% layout settings
\renewcommand{\footrulewidth}{0.5pt}
%\setlength\parindent{0pt}
\setlength{\columnsep}{3cm}

% listings options
\lstset{
  basicstyle=\itshape,
  xleftmargin=3em,
  literate={->}{$\rightarrow$}{2}
           {lambda}{$\lambda$}{1}
           {epsilon}{$\varepsilon$}{1}
           {times}{$\times$}{1}
           {start}{$S_{0}$}{1}
}

% header settings
\pagestyle{fancy}
\lhead{AI-NAND: Heuristic Analysis}
\rhead{Gerald Brown}
\cfoot{}
\rfoot{\thepage}
\begin{document}
\section*{Approach}
With my approach to providing an evaluation function, I iterated over a
multitude of different approaches. After the initial batch of testing, more
simplistic evaluation functions outperformed the more comprehensive evaluation
functions and were subsequently scrapped. The three evaluation
functions \code{AB_Custom()}, \code{AB_Custom_2()}, and \code{AB_Custom_3()} 
are the result of improving upon the simpler heuristics from that initial round
of testing.

\begin{codeblock}{improved\_score() from sample\_players.py}
    if game.is_loser(player):
        return float("-inf")
    if game.is_winner(player):
        return float("inf")

    own_moves = len(game.get_legal_moves(player))
    opp_moves = len(game.get_legal_moves(game.get_opponent(player)))
    return float(own_moves - opp_moves)
\end{codeblock}
    
Each evaluation functions starts with \code{improved_score()} as
a foundation and builds upon it by including additional variables based on
features of the game state and modifiers to said variables (in the form
of constants, exponents, etc.). This rationale stemmed from the poor
performance of more comprehensive evaluation functions when tested against
\code{improved_score()}. This revealed the computational trade-offs were
not worth it, in this case, contributing to the emphasis of simplicity in
further development of evalution functions.

% tertiary moves
Notable for its usage in two of the three custom evaluation functions are
``tertiary moves''. We define tertiary moves as the set of legal moves $A_L^\prime$
available to our player in a gamestate $s'$ after applying an action $a\in A_L$ 
to a gamestate $s$, where $a$ is a move in the set $A_L$ obtained from calling the
function \code{game.get_legal_moves(player)}.

\begin{codeblock}{get\_tertiary\_moves(game, legal\_moves) from game\_agent.py}
MOVEMENT_VECTORS = [(1, -2), (1, 2), (-2, -1), (-2, 1),
        (-1, -2), (-1, 2), (2, -1), (2, 1)]

def get_tertiary_moves(game, legal_moves):
    tertiary_moves = {(lm_y + v_y, lm_x + v_x)
                      for v_y, v_x in MOVEMENT_VECTORS
                      for lm_y, lm_x in legal_moves
                      if game.move_is_legal((lm_y + v_y, lm_x + v_x))}
    return tertiary_moves
\end{codeblock}

The set of tertiary moves is determined by applying a set of vectors 
representing the legal movement of a Knight to the player's position in
$s^\prime$ and determining if that is valid. The validity is dependent solely
upon whether the position is within the dimensions of the game board, and if
the position is vacant -- it does not account for the possiblity of the
opponent moving to that position following $s^\prime$.

\section*{Implementation}

\begin{codeblock}{all custom\_score() implementations include the following}
    if game.is_winner(player):
        return float("inf")
    if game.is_loser(player):
        return float("-inf")
\end{codeblock}

As previously mentioned, each heuristic treats \code{improved_score()} as an
initial starting point. With that, the above snippet is common to each of them
meaning each treats terminal game states in the same way.

\begin{codeblock}{custom\_score(game, player) from game\_agent.py}
    player_pos      =  game.get_player_location(player)
    legal_moves     =  game.get_legal_moves(player)
    tertiary_moves  =  get_tertiary_moves(game, legal_moves)
    opponent        =  game.get_opponent(player)
    opponent_pos    =  game.get_player_location(opponent)
    opponent_moves  =  game.get_legal_moves(opponent)
    return (len(legal_moves)
            + 0.5 * float(len(tertiary_moves)
                  / get_manhattan_distance(player_pos, opponent_pos))
            - len(opponent_moves))
\end{codeblock}

With \code{custom_score()}, we have the most obvious case from which
\code{improved_score()} served as the foundation. This heuristic is identical
to \code{improved_score()}, while the number of tertiary moves grows in
importance as the player and its opponent near each other. The constant
\code{0.5} was determined through experimentation.

\begin{codeblock}{custom\_score\_2(game, player) from game\_agent.py}
    opponent        =  game.get_opponent(player)
    player_pos      =  game.get_player_location(player)
    opponent_pos    =  game.get_player_location(opponent)

    return float(len(game.get_legal_moves(player))**2
                 / (get_manhattan_distance(player_pos, opponent_pos)
                    + len(game.get_legal_moves(opponent))**2))
\end{codeblock}

\code{custom_score_2()} works similarly but maintains an inverse
relationship between the number of legal moves available to the player and the 
opponent. This heuristic servers as an attempt at maintaining proximity
to the opponent after testing indicated the failings of other heuristics
against the centering heuristic. Tertiary moves are not included so as to save
computation time for iterative deepening.

\begin{codeblock}{custom\_score\_3(game, player) from game\_agent.py}
    player_pos      =  game.get_player_location(player)
    legal_moves     =  game.get_legal_moves(player)
    tertiary_moves  =  get_tertiary_moves(game, legal_moves)
    opponent        =  game.get_opponent(player)
    opponent_pos    =  game.get_player_location(opponent)
    return (float((0.5 * len(legal_moves) + len(tertiary_moves))
                  / get_manhattan_distance(player_pos, opponent_pos))
            - len(game.get_legal_moves(opponent)))
\end{codeblock}

With \code{custom_score_3()}, tertiary moves are given precedence over the set
of legal moves. The constant of \code{0.5} was arrived at through testing.

\section*{Results}

Testing revealed inconsistent results hence the additional rounds, as well as
incremental matches per round, of testing. Testing was initially limited to
opponents utilizing alpha-beta pruning, with the exception of the random agent.
Mini-max agents were added back in for the sake of thoroughness during what was
intended to be a final round of testing, only to reveal further inconsistencies
in terms of the winningest evaluation heuristic.
\begin{figure}[H]
\begin{Verbatim}[fontsize=\footnotesize]
  **********************************************************************
  *                              LEGEND                                *
  *   (of AB_Custom, AB_Custom_2, AB_Custom_3 wins against Opponent)   *
  *   --------------------------------------------------------------   *
  *                        *: 1st most wins                            *
  *                        +: 2nd most wins                            *
  **********************************************************************

                        *************************                         
                           Playing 150 Matches                              
                        *************************                         
 
                                 Round 1 

 Match #   Opponent    AB_Improved   AB_Custom   AB_Custom_2  AB_Custom_3 
                        Won | Lost   Won | Lost   Won | Lost   Won | Lost 
    1       Random      153 |   7    149 |  11   +151 |   9   *155 |   5  
    2       AB_Open     85  |  75   *95  |  65   +81  |  79    78  |  82  
    3      AB_Center    94  |  66   +94  |  66   *101 |  59    93  |  67  
    4     AB_Improved   81  |  79   *88  |  72   +84  |  76    78  |  82  
--------------------------------------------------------------------------
           Win Rate:      64.5%     *  66.6%     +  65.2%        63.1%    
\end{Verbatim}
\end{figure}

The first round of testing revealed that \code{custom_score()} was the most
winningest evaluation heuristic, on average. With considering the results of
all but those against the Random opponent, \code{AB_Custom} exhibits an even
higher win rate than \code{AB_Custom_2} and \code{AB_Custom_3}.

\begin{figure}[H]
\begin{Verbatim}[fontsize=\footnotesize]
                        *************************                         
                           Playing 200 Matches                              
                        *************************                         
                                 
                                 Round 2                             
 
 Match #   Opponent    AB_Improved   AB_Custom   AB_Custom_2  AB_Custom_3 
                        Won | Lost   Won | Lost   Won | Lost   Won | Lost 
    1       Random      187 |  13   +183 |  17   +183 |  17   *191 |   9  
    2       AB_Open     110 |  90   +105 |  95   *114 |  86    97  |  103 
    3      AB_Center    114 |  86    113 |  87   *117 |  83   +116 |  84  
    4     AB_Improved   104 |  96   *102 |  98   +99  |  101   98  |  102 
--------------------------------------------------------------------------
           Win Rate:      64.4%     +  62.9%     *  64.1%        62.8%  

                                 Round 3                             

 Match #   Opponent    AB_Improved   AB_Custom   AB_Custom_2  AB_Custom_3 
                        Won | Lost   Won | Lost   Won | Lost   Won | Lost 
    1       Random      187 |  13   +187 |  13   +187 |  13   *190 |  10  
    2       AB_Open     107 |  93    99  |  101  +101 |  99   *109 |  91  
    3      AB_Center    118 |  82   *122 |  78    116 |  84   +117 |  83  
    4     AB_Improved   99  |  101  +99  |  101  *101 |  99   +99  |  101 
--------------------------------------------------------------------------
           Win Rate:      63.9%     +  63.4%        63.1%     *  64.4%    

                                 Round 4                             

 Match #   Opponent    AB_Improved   AB_Custom   AB_Custom_2  AB_Custom_3 
                        Won | Lost   Won | Lost   Won | Lost   Won | Lost 
    1       Random      193 |   7    182 |  18   *193 |   7   +185 |  15  
    2       MM_Open     158 |  42   +150 |  50    148 |  52   *157 |  43  
    3      MM_Center    177 |  23    175 |  25   *179 |  21   +176 |  24  
    4     MM_Improved   144 |  56   *150 |  50   +148 |  52    140 |  60  
    5       AB_Open     104 |  96    97  |  103  *107 |  93   +102 |  98  
    6      AB_Center    121 |  79   *119 |  81   +117 |  83    115 |  85  
    7     AB_Improved   94  |  106  *112 |  88    103 |  97   +104 |  96  
--------------------------------------------------------------------------
           Win Rate:      70.8%     +  70.4%     *  71.1%        69.9%    

--------------------------------------------------------------------------
--------------------------------------------------------------------------
       Avg Win Rate
        over Rounds:                   65.6%     *  66.1%      + 65.7%    
\end{Verbatim}
\end{figure}

With increasing the number of matches played per round, and reintroducing
the mini-max tree traversal, \code{AB_Custom_2} took the lead. Examining
the results of each round, however, appears to delegitimize these results as
indicative of the efficacy of the evaluation heuristics and, instead, to 
be the result of favorable RNG outcomes. Due to this, the number of matches was
increased further for a final 5th round.

\begin{figure}[H]
\begin{Verbatim}[fontsize=\footnotesize]
                        *************************                         
                           Playing 1000 Matches                              
                        *************************                         

                                 Round 5

 Match #   Opponent    AB_Improved   AB_Custom   AB_Custom_2  AB_Custom_3 
                        Won | Lost   Won | Lost   Won | Lost   Won | Lost 
    1       Random      929 |  71    933 |  67   *937 |  63   +934 |  66  
    2       MM_Open     759 |  241  *778 |  222   752 |  248  +755 |  245 
    3      MM_Center    862 |  138   860 |  140  +870 |  130  *871 |  129 
    4     MM_Improved   726 |  274  *760 |  240  +734 |  266   728 |  272 
    5       AB_Open     525 |  475   542 |  458  *558 |  442  +550 |  450 
    6      AB_Center    558 |  442  +561 |  439   558 |  442  *565 |  435 
    7     AB_Improved   482 |  518  +503 |  497   489 |  511  *518 |  482 
--------------------------------------------------------------------------
           Win Rate:      69.2%     *  70.5%        70.0%     +  70.3%    

--------------------------------------------------------------------------
--------------------------------------------------------------------------
           Win Rate:                *  66.7%        66.0%     +  66.4%
      (excluding Random)
\end{Verbatim}
\end{figure}

Surprisingly, all custom evaluation heuristics performed better than
\code{AB_Improved}, albeit to a minor degree. Isolating the results of this
final round, \code{AB_Custom} exhibits the highest win rate. 
\section*{Selection}
Based upon the results of Round 5, as well the win rates when excluding the 
Random agent -- as I would not expect randomization to be the sole algorithm of
any effective evaluation heuristic -- I recommend the usage of the 
\code{AB_Custom} evaluation heuristic. This is due to its overall performance
(its win rate) during Round 5's testing, it's win rate with excluding the
Random agent for the rationale cited above, as well as it consistently either
having the highest or second highest win rate over all five rounds of testing.
\end{document}
\grid
